\documentclass[review]{igs}
\usepackage[utf8]{inputenc}
\usepackage{igsnatbib}
\usepackage{stfloats}
\usepackage{graphicx}
\graphicspath{{figures/}}

% ----------------------------------------------------------------------

\begin{document}

\title[Temperature variability in a positive degree day model]{Spatial and seasonal effects of temperature variability in a positive degree day surface melt model}
\author[J. Seguinot]{Julien Seguinot}
\affiliation{Department of Physical Geography and Quaternary Geology, Stockholm University, S-106 91 Stockholm, Sweden}

\maketitle

% ----------------------------------------------------------------------

\section{Introduction}

The positive degree day model is a parametrization of surface melt introduced by \citet{braithwaite-1984} and widely used in glacier modelling. Melt is assumed to be proportional to the number of Positive Degree Days (PDD), defined as the integral of positive, daily mean Celcius temperature $T$ over a time interval $A$, typically a year,

\begin{equation} \label{eq:pdd}
  \mathrm{PDD} = \int_{0}^{A}\max(T(t),0)\,dt.
\end{equation}

The effect of day-to-day variations of temperature on surface melt was recognized by \citet{braithwaite-1984} and later modelled by \citet{reeh-1991}, who assumed a normal probability distribution of $T$ around its annual cycle $T_{ac}$ and rewrote the PDD as

\begin{equation} \label{eq:reeh}
  \mathrm{PDD} = \frac{1}{\sigma\sqrt{2\pi}}
    \int_{0}^{A} \mathrm{d}t
    \int_{0}^{\infty} \mathrm{d}T \,
    T \exp\left({-\frac{(T-T_{ac}(t))^2}{2\sigma^2}}\right),
\end{equation}

where $\sigma$ is the standard deviation of the temperature distribution. More recently, \citet{calov-greve-2005} reformulated this equation to provide a more computationally-efficient expression for use in numerical glacier models,

\begin{equation} \label{eq:calovgreve}
  \mathrm{PDD} = \int_{0}^{A} \left[
    \frac{\sigma}{\sqrt{2\pi}}
    \exp\left({-\frac{T_{ac}(t)^2}{2\sigma^2}}\right)
    +\frac{T_{ac}(t)}{2} \,
    \mathrm{erfc} \left(-\frac{T_{ac}(t)}{\sqrt{2}\sigma}\right)
  \right]dt.
\end{equation}

The normal distribution approach has been implemented and used in several glacier models \citep[e.g.][]{letreguilly-etal-1991,greve-1997,seddik-etal-2012,charbit-etal-2013}. However, $\sigma$ is often assumed constant in time and space, despite being a major constraint on modelled surface melt and subsequent ice sheet geometries \citep{charbit-etal-2013}. Here I show that $\sigma$ is in fact highly variable spatially and seasonally, which has significant effects on the PDD computation.

% ----------------------------------------------------------------------

\section{Temperature variability}

The monthly standard deviation of daily mean temperature was computed from the ERA-Interim reanalysis \citep{data:erai} data using the following algorithm:

\begin{enumerate}
  \item daily means of surface air temperature were computed as averages of the four daily analysis time steps (00:00, 06:00, 12:00, 18:00);
  \item daily means were stacked by month over a 24-year (1979--2012) time interval;
  \item twelve monthly standard deviation grids were computed from the stacked daily means.
\end{enumerate}

\begin{figure}
  \centering\includegraphics{stdev}
  \caption{January (top) and July (bottom) distributions of monthly standard deviation of daily mean surface air temperature from the ERA-Interim reanalysis \citep{data:erai} data for the period 1979--2012. The maps show higher values in winter and over continental regions.}
  \label{fig:stdev}
\end{figure}

This approach captures the day-to-day and year-to-year variations of surface air temperature. However, the annual temperature cycle has not been removed from the time-series, which may have lead to slightly overestimated values in spring and autumn when temperature varies significantly between the beginning and the end of a month. Computed standard devations range from 0.32 to 12.63\,K, with a global annual average of 2.02\,K. As shown in Figure~\ref{fig:stdev}, there is a tendency for higher values over continental regions and in winter.

% ----------------------------------------------------------------------

\section{PDD modelling}

A reference PDD distribution, hereafter $\mathrm{PDD_{ERA}}$, was computed using equation~\ref{eq:calovgreve}, ERA-Interim monthly mean of daily mean surface air temperature, and ERA-Interim monthly standard deviation of daily mean surface air temperature described above. Furthermore, a PDD distribution was computed in four additional scenarios:

\begin{enumerate}
  \item using equation~\ref{eq:pdd} ($\mathrm{PDD_{0}}$);
  \item using equation~\ref{eq:calovgreve} with $\sigma=5$ ($\mathrm{PDD_{5}}$);
  \item using an annual mean of standard deviation ($\mathrm{PDD_{ANN}}$);
  \item using a boreal summer (JJA) mean of standard deviation ($\mathrm{PDD_{JJA}}$).
\end{enumerate}

For each scenario, absolute and relative errors relative to $\mathrm{PDD_{ERA}}$ were computed as

\begin{eqnarray}
  d_i &=& \mathrm{PDD}_i - \mathrm{PDD_{ERA}}, \mathrm{and}\\
  r_i &=& \left|\frac{d_i}{\mathrm{PDD_{ERA}}}\right|.
\end{eqnarray}

\begin{figure}
  \centering\includegraphics{adiff}
  \caption{Absolute PDD error using (a) zero temperature variability, (b) fixed temperature variability ($\sigma=5\,\mathrm{K}$), (c) mean annual temperature variability, (d) mean JJA temperature variability.}
  \label{fig:adiff}
\end{figure}

\begin{figure}
  \centering\includegraphics{rdiff}
  \caption{Relative PDD error in the same scenarios as in Figure~\ref{fig:adiff}.}
  \label{fig:rdiff}
\end{figure}

All four scenarios lead to significant errors in PDD estimates when applied on a continental scale (Figures~\ref{fig:adiff} and ~\ref{fig:rdiff}). Assuming zero temperature variability generally underestimates PDD values. Assuming $\sigma=5$ reduces errors in continental northern America and Eurasia but overestimates PDD values in coastal locations and over the Greenland ice sheet. Using mean annual standard deviation overestimates melt in most of the Arctic. Using mean JJA standard deviation generally yieldes smaller errors; however significant relative errors remain, particularly over the Greenland ice sheet where temperatures rarely rise above zero.

% ----------------------------------------------------------------------

\section{Conclusion}

The standard deviation of daily mean temperatures is highly variable both spatially and seasonally, and this variability significantly affects the accuracy of the PDD estimation. Numerical glacier models that use a PDD scheme should implement spatially and seasonally variable standard deviations of mean daily surface temperature in order to more realistically capture patterns of surface melt.

% ----------------------------------------------------------------------

\section{Acknowledgments}

I thank Caroline Clason, Ping Fu, Christian Helanow, Irina Rogozhina, Arjen Stroeven and Qiong Zhang for improving the quality and the clarity of this manuscript. Partial funding was provided by the Swedish Research Council (VR) to Stroeven (No. 2008-3449) and Stockholm University.

\bibliography{references}
\bibliographystyle{igs}
\end{document}

