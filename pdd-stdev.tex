\documentclass[twocolumn]{igs}
\usepackage{igsnatbib}
\usepackage{stfloats}
\usepackage{graphicx}
\graphicspath{{figures/}}

% ----------------------------------------------------------------------

\begin{document}

\title[Temperature variability in a positive degree day model]{Distributed effects of temperature variability in a positive degree day mass-balance model}
\author[J. Seguinot]{Julien Seguinot}
\affiliation{Department of Physical Geography and Quaternary Geology, Stockholm University, S-106 91 Stockholm, Sweden}

\maketitle

% ----------------------------------------------------------------------

\section{Introduction}

The positive degree day model is a parametrization of surface melt introduced by \citet{braithwaite-1984} and widely used in glacier modelling. Surface melt is assumed to be proportional to the number of Positive Degree Day (PDD), defined as the integral of positive, daily Celcius temperatures over time,

\begin{equation} \label{eq:pdd}
  \mathrm{PDD} = \int_{0}^{A}\max(T(t),0)\,dt.
\end{equation}

The effect of day-to-day variations of temperature on surface melt was recognized by \citet{braithwaite-1984} and later modelled by \citet{reeh-1991}, assuming a normal probability distribution of temperature $T$ around a central value $T_{ac}$,

\begin{equation} \label{eq:normal}
  p(T) = \frac{1}{\sigma\sqrt{2\pi}}
    \exp\left({-\frac{(T-T_{ac})^2}{2\sigma^2}}\right).
\end{equation}

More recently, \citet{calov-greve-2005} showed that, under the normal distribution assumption, the PDD integral could then be reformulated as

\begin{equation} \label{eq:calovgreve}
  \mathrm{PDD} = \int_{0}^{A} \left[
    \frac{\sigma}{\sqrt{2\pi}}
    \exp\left({-\frac{T_{ac}(t)^2}{2\sigma^2}}\right)
    +\frac{T_{ac}(t)}{2}
    \mathrm{erfc} \left(-\frac{T_{ac}(t)}{\sqrt{2}\sigma}\right)
  \right]dt,
\end{equation}

where $T_{ac}$ is the annual temperature cycle and $\sigma$ and the standard deviation of temperature, thereby providing a more computationally-efficient expression to numerical glacier models.

This approach has now been implemented and used in several numerical glacier models, and the role of temperature variability has been emphasized by \citet{charbit-etal-2012}. However the $\sigma$ parameter if generally set as a constant in time and and space, which is a limitation to the accuracy of modelled mass balances \citep{charbit-etal-2012}.

Here I show that the $\sigma$ parameter is highly variable spatially and seasonaly, which has significant effects on PDD-modelled mass-balances.é

% ----------------------------------------------------------------------

\section{Temperature variability}

Monthly standard deviations of daily temperatures were computed from the ERA-Interim\citep{data:erai} analysis (step~0) data using the following algorythm:

\begin{enumerate}
  \item daily means of surface air temperature were calculated by averaging the four daily analysis time steps (00:00, 06:00, 12:00, 18:00);
  \item daily means were stacked by month over a 22-years (1979--2012) time interval;
  \item monthly standard deviation were computed for each month.
\end{enumerate}

This approach captures not only the day-to-day but also the year-to-year variation of air surface temperature. Temperature time-series, however, were not been detrended from the annual cycle, which may have lead to slightly overestimates values in Spring and Autumn months.

\begin{figure}
  \centering\includegraphics{stdev}
  \caption{January (top) and July (bottom) monthly standard deviations of daily mean temperature from the ERA-Interim analysis data. The maps show higher values in winter and in continental regions.}
  \label{fig:stdev}
\end{figure}

Computed values range from 0.32 to 12.63\,K, with a global seasonal average of 2.02\,K. January and July distributions are shown in Figure~\ref{fig:stdev}. There is a tendency for higher dtandard deviation of temperature in continental regions, and in winter.

% ----------------------------------------------------------------------

\section{PDD modelling}

The PDD number was computed in five cases:

\begin{enumerate}
  \item using equation~\ref{eq:calovgreve} with spatially-variable monthly values of standard deviation described above, noted $\mathrm{PDD_{ERA}}$,
  \item using equation~\ref{eq:pdd} (i.e. $\sigma=0$), noted $\mathrm{PDD_{0}}$,
  \item using equation~\ref{eq:calovgreve} with $\sigma=5$, noted $\mathrm{PDD_{5}}$
  \item using equation~\ref{eq:calovgreve} with annual averages of the spatially-variable standard deviation, noted $\mathrm{PDD_{ANN}}$,
  \item using equation~\ref{eq:calovgreve} with summer (JJA) averages of spatially-variable standard deviations, noted $\mathrm{PDD_{JJA}}$.
\end{enumerate}

The first case outlined aboved was used as a reference to compute absolute and relative errors on all other cases.

\begin{eqnarray}
  d_i &=& PDD_i - PDD_{ERA}, \\
  r_i &=& \left|\frac{d_i}{PDD_{ERA}}\right|.
\end{eqnarray}

\begin{figure}
  \centering\includegraphics{adiff}
  \caption{Absolute errors in PDD-modelled surface mass balance using (a) no temperature variability ($\sigma=0$), (b) fixed temperature variability ($\sigma=5\,\mathrm{K}$, (c) annually-average temperature variability, (d) summer-averaged temperature variability.}
  \label{fig:adiff}
\end{figure}

\begin{figure}
  \centering\includegraphics{rdiff}
  \caption{Absolute errors in PDD-modelled surface mass balance using similar cases as in Figure~\ref{fig:adiff}}
  \label{fig:rdiff}
\end{figure}

It appears that all four simplifying approaches lead to significant errors in PDD estimates when applied regionally (Figures~\ref{fig:adiff} and ~\ref{fig:rdiff}). Neglecting temperature variability ($\sigma=0$) leads to generally underestimated PDD values. Using $\sigma=5$ improves gives better results in continental, unglaciated regions but to overestimates near the coasts. Using annual means of standard deviations improves things regionally but greatly overestimates PDD in glaciated regions. Using the summer average gives more accurate results in glaciated regions, yet at the deficit of more temperate places.

These results suggests that spatially-variables, seasonally-variables values of the standard deviation should be used in order to capture accurate patterns of melt.

\bibliography{references}
\bibliographystyle{igs}
\end{document}

