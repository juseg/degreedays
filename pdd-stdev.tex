\documentclass[twocolumn]{igs}
\usepackage{igsnatbib}
\usepackage{stfloats}
\usepackage{graphicx}
\graphicspath{{figures/}}

% ----------------------------------------------------------------------

\begin{document}

\title[Temperature variability in a positive degree day model]{Distributed effects of temperature variability in a positive degree day mass-balance model}
\author[J. Seguinot]{Julien Seguinot}
\affiliation{Department of Physical Geography and Quaternary Geology, Stockholm University, S-106 91 Stockholm, Sweden}

\maketitle

% ----------------------------------------------------------------------

\section{Introduction}

The positive degree day model is a parametrization of surface melt introduced by \citet{braithwaite-1984} and widely used in glacier modelling. Melt is assumed to be proportional to the number of Positive Degree Day (PDD), defined as the integral of positive, daily Celcius temperatures over a time interval $A$ of typically one year,

\begin{equation} \label{eq:pdd}
  \mathrm{PDD} = \int_{0}^{A}\max(T(t),0)\,dt,
\end{equation}

The effect of day-to-day variations of temperature on surface melt was recognized by \citet{braithwaite-1984} and later modelled by \citet{reeh-1991}, who assumed a normal probability distribution of temperature $T$ around the annual cycle $T_{ac}$ and rewrote the PDD integral as

\begin{equation} \label{eq:reeh}
  \mathrm{PDD} = \frac{1}{\sigma\sqrt{2\pi}}
    \int_{0}^{A} \mathrm{d}t
    \int_{0}^{\infty} \mathrm{d}T \,
    T \exp\left({-\frac{(T-T_{ac}(t))^2}{2\sigma^2}}\right),
\end{equation}

where $\sigma$ is the standard deviation of the temperature distribution. More recently, \citet{calov-greve-2005} reformulated this equation to provide a more computationally-efficient expression to numerical glacier models,

\begin{equation} \label{eq:calovgreve}
  \mathrm{PDD} = \int_{0}^{A} \left[
    \frac{\sigma}{\sqrt{2\pi}}
    \exp\left({-\frac{T_{ac}(t)^2}{2\sigma^2}}\right)
    +\frac{T_{ac}(t)}{2} \,
    \mathrm{erfc} \left(-\frac{T_{ac}(t)}{\sqrt{2}\sigma}\right)
  \right]dt.
\end{equation}

Their approach has now been implemented and used in several glacier models, and the role of temperature variability has been emphasized by \citet{charbit-etal-2012}. However the $\sigma$ parameter is generally set constant in time and space, which is a limitation to the accuracy of modelled mass balances \citep{charbit-etal-2012}.

Here I show that the $\sigma$ parameter is highly variable spatially and seasonally, which has significant effects on the PDD computation.

% ----------------------------------------------------------------------

\section{Temperature variability}

Monthly standard deviations of daily temperatures were computed from the ERA-Interim \citep{data:erai} analysis data using the following algorithm:

\begin{enumerate}
  \item daily means of surface air temperature were calculated by averaging the four daily analysis time steps (00:00, 06:00, 12:00, 18:00);
  \item daily means were stacked by month over a 22-year (1979--2012) time interval;
  \item a standard deviation was computed for each month.
\end{enumerate}

This approach captures not only day-to-day but also year-to-year variations of air surface temperature. Temperature time-series, however, have not been de-trended from the annual cycle, which may have lead to slightly overestimated values in Spring and Autumn.

\begin{figure}
  \centering\includegraphics{stdev}
  \caption{January (top) and July (bottom) monthly standard deviations of daily mean temperature from the ERA-Interim analysis data. The maps show higher values in winter and in continental regions.}
  \label{fig:stdev}
\end{figure}

Computed values range from 0.32 to 12.63\,K, with a global seasonal average of 2.02\,K. January and July distributions are shown in Figure~\ref{fig:stdev}. There is a tendency for higher values in continental regions and in Winter.

% ----------------------------------------------------------------------

\section{PDD modelling}

A reference PDD distribution, hereafter $\mathrm{PDD_{ERA}}$, was computed from equation~\ref{eq:calovgreve}, using ERA-Interim monthly averages of daily temperature and the monthly values of standard deviations described above.

Furthermore, PDDs were computed in four simplifying cases:

\begin{enumerate}
  \item using equation~\ref{eq:pdd} ($\mathrm{PDD_{0}}$),
  \item using equation~\ref{eq:calovgreve} with $\sigma=5$ ($\mathrm{PDD_{5}}$),
  \item using annually-averaged standard deviations ($\mathrm{PDD_{ANN}}$),
  \item using summer (JJA)-averaged standard deviations ($\mathrm{PDD_{JJA}}$).
\end{enumerate}

For each case, absolute and relative errors were computed as

\begin{eqnarray}
  d_i &=& \mathrm{PDD}_i - \mathrm{PDD_{ERA}}, \\
  r_i &=& \left|\frac{d_i}{\mathrm{PDD_{ERA}}}\right|.
\end{eqnarray}

\begin{figure}
  \centering\includegraphics{adiff}
  \caption{Absolute errors in PDD-modelled surface mass balance using (a) no temperature variability ($\sigma=0$), (b) fixed temperature variability ($\sigma=5\,\mathrm{K}$, (c) annually-average temperature variability, (d) summer-averaged temperature variability.}
  \label{fig:adiff}
\end{figure}

\begin{figure}
  \centering\includegraphics{rdiff}
  \caption{Relative errors in PDD-modelled surface mass balance using similar cases as in Figure~\ref{fig:adiff}}
  \label{fig:rdiff}
\end{figure}

All four simplifying approaches lead to significant errors in PDD estimates when applied regionally (Figures~\ref{fig:adiff} and ~\ref{fig:rdiff}). Neglecting temperature variability ($\sigma=0$) leads to generally underestimated PDD values. Assuming $\sigma=5$ reduces errors in continental, temperate regions but overestimates melt near the coasts. Using an annually-averaged standard deviation overestimates melt in most of the Arctic. Using a summer average gives more accurate results in colder regions, however significant errors remain.

\section{Conclusion}

The standard deviation of daily temperatures is highly variable both spatially and seasonally, and this variability significantly affects the PDD calculation. Numerical glacier models that use a PDD scheme should implement spatially and seasonally variable standard deviations of temperature in order to better capture patterns of surface melt.

\bibliography{references}
\bibliographystyle{igs}
\end{document}

