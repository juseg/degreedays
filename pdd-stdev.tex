\documentclass[twocolumn]{igs}
\usepackage{igsnatbib}
\usepackage{stfloats}

% ----------------------------------------------------------------------

\begin{document}

\title[Temperature variability in a positive degree day model]{Distributed effects of temperature variability in a positive degree day mass-balance model}
\author[J. Seguinot]{Julien Seguinot}
\affiliation{Department of Physical Geography and Quaternary Geology, Stockholm University, S-106 91 Stockholm, Sweden}

\maketitle

% ----------------------------------------------------------------------

\section{Introduction}

The positive degree day model is a parametrization of surface melt introduced by \citet{braithwaite-1984} and widely used in glacier modelling. Surface melt is assumed to be proportional to the number of Positive Degree Day (PDD), defined as the integral of positive, daily Celcius temperatures over time.

The effect of day-to-day variations of temperature on surface melt was recognized by \citet{braithwaite-1984} and later modelled by \citet{reeh-1991}, assuming a normal probability distribution of temperature $T$ around a central value $T_{ac}$,

\begin{equation}
  p(T) = \frac{1}{\sigma\sqrt{2\pi}}
    \exp\left({-\frac{(T-T_{ac})^2}{2\sigma^2}}\right).
\end{equation}

More recently, \citet{calov-greve-2005} showed that, under the normal distribution assumption, the PDD integral could then be reformulated as

		\exp\left({-\frac{T_{ac}(t)^2}{2\sigma^2}}\right)
		+\frac{T_{ac}(t)}{2}
		\mathrm{erfc} \left(-\frac{T_{ac}(t)}{\sqrt{2}\sigma}\right)
	\right],
\end{equation}

where $T_{ac}$ is the annual temperature cycle and $\sigma$ and the standard deviation of temperature, thereby providing a more computationally-efficient expression to numerical glacier models.

This approach has now been implemented and used in several numerical glacier models, and the role of temperature variability has been emphasized by \citet{charbit-etal-2012}. However the $\sigma$ parameter if generally set as a constant in time and and space, which is a limitation to the accuracy of modelled mass balances \citep{charbit-etal-2012}.

Here I show that the $\sigma$ parameter is highly variable spatially and seasonaly, which has significant effects on PDD-modelled mass-balances.é

\section{Temperature variability}

\section{Surface mass balance}

\bibliography{references}
\bibliographystyle{igs}
\end{document}

