\documentclass[twocolumn]{igs}
\usepackage[utf8]{inputenc}
\usepackage{igsnatbib}
\usepackage{stfloats}
\usepackage{graphicx}
\graphicspath{{figures/}}

% ----------------------------------------------------------------------

\begin{document}

\title[Correspondence]{Correspondence}
\author[J. Seguinot]{Julien Seguinot}
\affiliation{Department of Physical Geography and Quaternary Geology, Stockholm University, S-106 91 Stockholm, Sweden}
\abstract{Empty}

\maketitle

% ----------------------------------------------------------------------

\emph{Spatial and seasonal effects of temperature variability in a positive degree day glacier surface mass balance model}

\section{Introduction}

The positive degree day model is a parametrization of surface melt widely used for its simplicity \citep{hock-2003}. Melt is assumed to be proportional to the number of Positive Degree Days (PDD), defined as the integral of positive Celcius temperature $T$ over a time interval $A$,

\begin{equation} \label{eq:pdd}
  \mathrm{PDD} = \int_{0}^{A}\max(T(t),0)\,dt.
\end{equation}

When modelling glaciers on the multi-millennium time-scale needed for spin-up simulations or palaeo-ice sheet reconstructions \citep[e.g.][]{charbit-etal-2013}, daily or hourly temperature data is usually not available and PDDs are typically calculated over one year using an average annual temperature cycle. Sub-annual temperature variability around the freezing point, however, significantly affects surface melt on a multi-year scale \citep{arnold-mackay-1964}. It is commonly included in models by assuming a normal probability distribution of $T$ of known standard deviation $\sigma$ around the annual cycle $T_{ac}$ \citep{braithwaite-1984}. PDDs can then be computed using a double-integral formulation of \citet{reeh-1991},

\begin{equation} \label{eq:reeh}
  \mathrm{PDD} = \frac{1}{\sigma\sqrt{2\pi}}
    \int_{0}^{A} \mathrm{d}t
    \int_{0}^{\infty} \mathrm{d}T \,
    T \exp\left({-\frac{(T-T_{ac}(t))^2}{2\sigma^2}}\right),
\end{equation}

or more efficiently using an error function formulation of \citet{calov-greve-2005},

\begin{eqnarray} \label{eq:calovgreve}
  \mathrm{PDD} = \int_{0}^{A} &\Big[ &
    \frac{\sigma}{\sqrt{2\pi}}
    \exp\left({-\frac{T_{ac}(t)^2}{2\sigma^2}}\right) \nonumber\\
    && + \frac{T_{ac}(t)}{2} \,
    \mathrm{erfc} \left(-\frac{T_{ac}(t)}{\sqrt{2}\sigma}\Big)
  \right]dt.
\end{eqnarray}

These approaches have been implemented and used in several glacier models \citep[e.g.][]{letreguilly-etal-1991,greve-1997,huybrechts-dewolde-1999,seddik-etal-2012,charbit-etal-2013}. However, with the exception of an available parametrization for the Greenland Ice Sheet \citep{fausto-etal-2011}, $\sigma$ is often assumed constant in time and space, despite its large influence on modelled surface melt and subsequent ice sheet geometries \citep{charbit-etal-2013}. Here I show that $\sigma$ is in fact highly variable spatially and seasonally, which has significant effects on PDD and Surface Mass Balance (SMB) computation.

% ----------------------------------------------------------------------

\section{Temperature variability}

Using the ERA-Interim reanalysis data \citep{data:erai} for the period 1979--2012, a monthly climatology is prepared, consisting of long-term monthly mean surface air temperature, long-term mean monthly precipitation and long-term monthly standard deviation of daily mean surface air temperature. Standard deviation is calculated using temperature deviations relative to the long-term monthly mean for the entire period, in order to capture both day-to-day and year-to-year variations of surface air temperature. Daily mean surface air temperature is computed beforehand as an average of the four daily analysis time steps (00:00, 06:00, 12:00, 18:00), to avoid variability associated with the diurnal cycle. Since the annual temperature cycle is not removed from the time-series, this approach may lead to slightly overestimated standard deviation values in spring and autumn when temperature varies significantly between the beginning and the end of a month. 

\begin{figure}
  \centering\includegraphics{stdev}
  \caption{Spatial distributions of standard deviation of daily mean surface air temperature for January (top) and July (bottom) based on the ERA-Interim reanalysis data for the period 1979--2012. The maps show higher values in winter and over continental regions \citep{data:erai}.}
  \label{fig:stdev}
\end{figure}

Computed standard deviations range from 0.32 to 12.63\,K, with a global annual average of 2.02\,K. There is a tendency for higher values over continental regions (Figure~\ref{fig:stdev}). Additionally, computed standard deviation follow an annual cycle with a winter high and a summer low. This extends the conclusions obtained by \citet{fausto-etal-2011} over Greenland to non-glaciated regions, where summer temperatures are not constrained by 0 $^\circ$C by an ice surface.

% ----------------------------------------------------------------------

\section{Positive degree days}

A reference PDD distribution, hereafter $\mathrm{PDD_{ERA}}$, is computed using equation~\ref{eq:calovgreve} over one year and the monthly climatology described above. Furthermore, annual PDDs are computed using four additional, simplifying scenarios:

\begin{enumerate}
  \item using equation~\ref{eq:pdd} ($\rm PDD_0$);
  \item using equation~\ref{eq:calovgreve} with $\sigma=5\,K$, a value commonly used in ice-sheet modelling \citep{huybrechts-dewolde-1999,seddik-etal-2012,charbit-etal-2013} ($\rm PDD_5$);
  \item using an annual mean of standard deviation ($\rm PDD_{ANN}$);
  \item using a boreal summer (JJA) mean of standard deviation ($\rm PDD_{JJA}$).
\end{enumerate}

\begin{figure}
  \centering\includegraphics{pdd-adiff-arctic}
  \caption{PDD error ($\Delta\mathrm{PDD}_i = \mathrm{PDD}_i - \mathrm{PDD_{ERA}}$) over the Arctic using (a) zero temperature variability, (b) fixed temperature variability ($\sigma=5\,\mathrm{K}$), (c) mean annual temperature variability, (d) mean summer temperature variability.}
  \label{fig:pdd}
\end{figure}

All four scenarios lead to significant errors in PDD estimates when applied on a continental scale (Figure~\ref{fig:pdd}). Assuming zero temperature variability ($\rm PDD_0$) generally underestimates PDD values. Assuming $\sigma=5\,K$ ($\rm PDD_5$) reduces errors in continental northern America and Eurasia but overestimates PDD values in coastal locations and over the Greenland Ice Sheet. Using annual mean ($\rm PDD_{ANN}$) or summer mean ($\rm PDD_{JJA}$) standard deviation generally yields smaller, yet significant errors.

% ----------------------------------------------------------------------

\section{Surface mass balance}

For each scenario, SMB is computed using a simple annual PDD model (github.com/jsegu/pypdd). Accumulation is assumed equal to precipitation when temperature is below 0\,$\rm^\circ C$, and decreasing linearly with temperature between 0 and 2\,$\rm^\circ C$. Melt is computed using degree day factors of 3\,$\rm mm\,^\circ C^{-1}\,d^{-1}$ for snow and 8\,$\rm mm\,^\circ C^{-1}\,d^{-1}$ for ice \citep{huybrechts-dewolde-1999}. Surface mass-balance is computed over week-long intervals prior to annual integration using a scheme similar to the one implemented in the ice sheet model PISM (www.pism-docs.org).

\begin{figure}
  \centering\includegraphics{smb-adiff-greenland}
  \caption{Surface mass balance error ($\Delta\mathrm{SMB}_i = \mathrm{SMB}_i - \mathrm{SMB_{ERA}}$) over Greenland using the same scenarios as in Figure~\ref{fig:pdd}.}
  \label{fig:smb}
\end{figure}

Over Greenland, using summer mean standard deviation in the SMB calculation ($\rm SMB_{JJA}$) gives smaller errors than other simplifying scenarios (Figure~\ref{fig:smb}). This confirms recent results obtained by \citet{rau-rogozhina-2013} from the ERA-40 reanalysis data. However in all four cases, large SMB errors occurs along the margin where most of the melt processes take place.

% ----------------------------------------------------------------------

\section{Conclusion}

Using gridded climate products, temperature variability can be assessed quantitatively on the continental scale. Monthly standard deviation of daily mean surface air temperature is highly variable both spatially and seasonally. When using standard deviation PDD formulations \citep{braithwaite-1984,reeh-1991,calov-greve-2005}, approximations of constant standard deviation do not hold on the continental scale and introduce significant errors in modelled PDD and SMB, which have immediate implications for numerical modelling studies of present-day and former glaciations. Under the assumption of a normal distribution of temperature around the annual cycle, numerical glacier models that use an annual PDD scheme should therefore implement spatially and seasonally variable standard deviations of daily mean surface air temperature in order to more realistically capture patterns of surface melt.

% ----------------------------------------------------------------------

\section{Acknowledgments}

I thank Caroline Clason, Ping Fu, Christian Helanow, Andrew Mercer, Arjen Stroeven and Qiong Zhang for their help in preparing the first version of this paper, Irina Rogozhina for multiple advice on text and figures, Constantine Khroulev for sharing details of PISM's PDD implementation through e-mail, and finally Regine Hock and Roger Braithwaite for their constructive reviews and suggestions for improvements. Partial funding was provided by the Swedish Research Council (VR) to Stroeven (No. 2008-3449) and Stockholm University.

\bibliography{references}
\bibliographystyle{igs}
\end{document}
