\documentclass[review]{igs}
\usepackage[utf8]{inputenc}
\usepackage{igsnatbib}
\usepackage{stfloats}
\usepackage{graphicx}
\graphicspath{{figures/}}

% ----------------------------------------------------------------------

\begin{document}

\title[Correspondence]{Correspondence}
\author[Seguinot and Rogozhina]{Julien Seguinot$^{1,2,3}$ and Irina Rogozhina$^1$}
\affiliation{%
    $^1$Helmholtz Centre Potsdam, GFZ German Research Centre for Geosciences,
    DE-14473 Potsdam, Germany\\
    $^2$Department of Physical Geography and Quaternary Geology,
    Stockholm University, SE-10691 Stockholm, Sweden\\
    $^3$Bolin Centre for Climate Research,
    Stockholm University, SE-10691 Stockholm, Sweden}
\abstract{Empty (no abstract for correspondence).}

\maketitle

% ----------------------------------------------------------------------

\emph{Daily temperature variability predetermined by thermal conditions over ice sheet surfaces}

Evolution of continental-scale ice sheets is affected by long-term changes in atmospheric conditions through shifts of the balance between surface accumulation and melt. Although surface melt primarily follows the annual temperature cycle, occasional excursions of air temperature above the freezing point caused by daily weather variability can significantly affect surface mass balance of ice sheets (ref.).

In multimillenial numerical simulations of ice sheet growth and decay, surface melt is commonly assumed proportional to the number of positive degree-days (PDD). Daily variability is then typically included in the models under an assumption of normal temperature distribution, using a standard deviation parameter $\sigma$ in the PDD computation;

\begin{equation} \label{eq:reeh}
  \mathrm{PDD} = \frac{1}{\sigma\sqrt{2\pi}}
    \int_{t_1}^{t_2} \mathrm{d}t
    \int_{0}^{\infty} \mathrm{d}T \,
    T \exp\left({-\frac{(T-T_a)^2}{2\sigma^2}}\right),
\end{equation}

where t is the time, $T$ is the air temperature [$^\circ$C], and $\sigma$ is the standard deviation of $T$ from the annual mean air temperature cycle $Ta$ [$^\circ$C].

It has been shown that this parameter is subject to significant variation over the globe, which predominates the surface mass balance model’s response (ref.). In Greenland, the analysis of in-situ measurements suggests that the present-day $\sigma$ distribution is subject to annual variability, with lowest values occurring in the melting period and at the ice sheet's margins, and can be linearly related to ice surface elevation (ref.). Recent modelling studies of the Greenland ice sheet show that low $\sigma$ values in the summer period and their spatial variation are key to a modelling of the present-day ice surface responses in agreement with observations (ref.). However, the important value of a PDD approach is its applicability to long-term palaeo studies, requiring realistic $\sigma$ values under climate conditions different from today. In this study, we present evidence from reanalysis data that $\sigma$ distribution in Greenland is mainly controlled by variation in near-surface air temperature and not elevation as previously thought.

Using the European Centre for Medium-Range Weather Forecasts’ ERA-40 reanalysis data (ref.) over the 44-years period 1958–2001, we compute the spatial distribution of long-term monthly mean surface air temperature $T$, and long-term monthly standard deviation of daily mean surface air temperature $\sigma$. Daily mean surface air temperature is computed in first hand as an average of the four daily analysis time-steps (00:00, 06:00, 12:00, 18:00). Standard deviation is calculated using temperature deviations relative to the long-term daily mean for the entire period, in order to remove variability associated with the annual cycle. Note that this is an improvement over our previous methods (refs.) that tended to overestimate spring and autumn $\sigma$ values.

\begin{figure*}
    \centering\includegraphics{stdev-param-scatter-sigma-era40-grl-all-large}
    \caption{Long-term monthly standard deviation compared to the long-term monthly mean of daily mean surface air temperature over the Greenland ice sheet, according to the ERA-40 reanalysis data (ref.) over the 44-years period 1958–2001. Seasons are coloured in red (JJA), yellow (SON), blue (DJF) and green (MAM). The solid line corresponds to a linear regression over all points. Dashed contours represent the effect of daily variability on effective temperature for melt, also shown in the 3D wireframe inset.}
    \label{fig:grl}
\end{figure*}

Over the Greenland ice sheet, there exists an anticorrelation between the long-term monthly temperature means and the long-term monthly standard deviations (Figure~\ref{fig:grl}). We derive a linear relation between T and $\sigma$,

\begin{equation}
    \sigma = 0.15 \cdot T + 2.01.
\end{equation}

Although the data shows a significant spread around winter temperatures (Figure~\ref{fig:grl}, blue), the reconstructed trend appears more robust over the period of intensive surface melting (Figure~\ref{fig:grl}, red), when the standard deviation parameter exerts strongest influence on modelled melt rates (ref.). To emphasize this we define, based on the integral formulation from Calov and Greve (2005), an effective temperature for melt,

\begin{eqnarray} \label{eq:calovgreve}
    T_e &=& \frac{d\mathrm{PDD}}{dt}\\
        &=& \frac{1}{\sigma\sqrt{2\pi}}
            \int_{0}^{\infty} \mathrm{d}T \, T \exp\left({-\frac{(T-T_a)^2}{2\sigma^2}}\right)\\
        &=& \frac{\sigma}{\sqrt{2\pi}} \exp\left({-\frac{T_a^2}{2\sigma^2}}\right)
            + \frac{T_a}{2} \, \mathrm{erfc} \left(-\frac{T_a}{\sqrt{2}\sigma}\right).
\end{eqnarray}

The net effect of daily variability on melt can then be expressed as a switch in effective temperature,

\begin{equation}
    \Delta T_e = T_e - \max(T_a, 0),
\end{equation}

and is displayed in Figure~\ref{fig:grl} (dashed contours). Note that this effect is always positive, hence the use of higher values systematically results in increased modelled surface melt (ref.). A decrease and associated restriction of ice melt in response to increasing air temperatures indicates a natural mechanism for self-inhibition of ice mass loss, which is, under the conditions close to the melting point, modulated by endothermic phase changes (Figure~\ref{fig:grl}).

On the one hand, a strong dependence of on air temperature shown by reanalysis data suggests that previously discussed relations between standard deviation and ice surface elevation (ref.) originate from slope lapse-rate effects on near-surface air temperature and are only valid under present-day climatic conditions. On the other hand, a nearly linear dependence of on temperature provides a simple recipe for a dynamic adjustment of the standard deviation parameter to evolving external forcing. One should note that the inferred relation is generally inapplicable to ice-free territories where more complex connections should be expected (Seguinot, 2013). Our analysis does however reveal a similar relation between air temperature and across Antarctica (Figure~\ref{fig:both}) and thus suggests potential applicability of our new parameterisation to modeling studies of glaciations in regions with essentially different geographical and climatic settings.

\begin{figure}
    \centering\includegraphics{stdev-param-scatter-sigma-era40-both-zoom-07}
    \caption{July Greenland (red) and January Antarctic (gray) long-term monthly standard deviation compared to the long-term monthly mean of daily mean surface air temperature over the ice sheets, according to the ERA-40 reanalysis data (ref.) over the 44-years period 1958–2001.}
    \label{fig:both}
\end{figure}

% ----------------------------------------------------------------------

\section{Acknowledgments}

DAAD, Ahlmann.

\bibliography{references}
\bibliographystyle{igs}

\end{document}

\citep{hock-2003}
\citep[e.g.][]{charbit-etal-2013}
Sub-annual temperature variability around the freezing point, however, significantly affects surface melt on a multi-year scale \citep{arnold-mackay-1964}.
\citep{braithwaite-1984}
\citet{reeh-1991},
\citet{calov-greve-2005},
\citep[e.g.][]{letreguilly-etal-1991,greve-1997,huybrechts-dewolde-1999,seddik-etal-2012,charbit-etal-2013}.
\citep{fausto-etal-2011}
\citep{charbit-etal-2013}
\citep{data:erai}
\citet{fausto-etal-2011}
