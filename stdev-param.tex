\documentclass[review]{igs}
\usepackage[utf8]{inputenc}
\usepackage{igsnatbib}
\usepackage{stfloats}
\usepackage{graphicx}
\graphicspath{{figures/}}

% ----------------------------------------------------------------------

\begin{document}

\title[Correspondence]{Correspondence}
\author[Seguinot and Rogozhina]{Julien Seguinot$^{1,2,3}$ and Irina Rogozhina$^1$}
\affiliation{%
    $^1$Helmholtz Centre Potsdam, GFZ German Research Centre for Geosciences,
    DE-14473 Potsdam, Germany\\
    $^2$Department of Physical Geography and Quaternary Geology,
    Stockholm University, SE-10691 Stockholm, Sweden\\
    $^3$Bolin Centre for Climate Research,
    Stockholm University, SE-10691 Stockholm, Sweden}
\abstract{Empty (no abstract for correspondence).}

\maketitle

% ----------------------------------------------------------------------

\emph{Daily temperature variability predetermined by thermal conditions over ice sheet surfaces}

~

Evolution of continental-scale ice sheets is affected by long-term changes in atmospheric conditions through shifts of the balance between surface accumulation and melt. Although surface melt primarily follows the annual temperature cycle, it is also significantly affected by occasional excursions of air temperature above the freezing point caused by daily weather variability \citep{arnold-mackay-1964}.

In multimillenial numerical simulations of ice sheet growth and decay, surface melt is commonly assumed proportional to the number of positive degree-days \citep[PDD;][]{hock-2003}. Daily variability is then typically included in the models under an assumption of normal temperature distribution, using a standard deviation parameter $\sigma$ in the PDD computation \citep{braithwaite-1984,reeh-1991,calov-greve-2005};

\begin{eqnarray} \label{eq:pdd}
    \mathrm{PDD} &=& \frac{1}{\sigma\sqrt{2\pi}}
        \int_{t_1}^{t_2} \mathrm{d}t
        \int_{0}^{\infty} \mathrm{d}T \,
        T \exp\left({-\frac{(T-T_a)^2}{2\sigma^2}}\right)\\
    &=& \int_{t_1}^{t_2} \mathrm{d}t
        \left[\frac{\sigma}{\sqrt{2\pi}} \exp\left({-\frac{T_a^2}{2\sigma^2}}\right)
        + \frac{T_a}{2} \, \mathrm{erfc} \left(-\frac{T_a}{\sqrt{2}\sigma}\right)\right],
\end{eqnarray}

where t is the time, $T$ is the air temperature [$^\circ$C], and $\sigma$ is the standard deviation of $T$ from the annual mean air temperature cycle $Ta$ [$^\circ$C]. According to atmospheric reanalysis data, $\sigma$ is subject to significant variation over the globe, which predominates the surface mass balance model’s response \citep{seguinot-2013}.

Over the Greenland ice sheet, the analysis of in-situ measurements suggests that the present-day $\sigma$ distribution is subject to annual variability, with lowest values occurring in the melting period and at the ice sheet's margins, and can be linearly related to ice surface elevation \citep{fausto-etal-2009,fausto-etal-2011}. Yet, including such seasonal and spatial $\sigma$ variability in simulations of the Greenland ice sheet is key to modelling the present-day ice surface response in agreement with observations \citep{rogozhina-rau-inpress}. However, the important value of a PDD approach is its applicability to studies of former glaciations, requiring realistic $\sigma$ values under climate conditions different from today. Here, we present evidence from reanalysis data that the $\sigma$ distribution over the Greenland ice sheet is largely related to variations in near-surface air temperature, which can provide the basis for a parametrization.

Using the European Centre for Medium-Range Weather Forecasts’ ERA-40 reanalysis data \citep{uppala-etal-2005} over the 44-years period 1958–2001, we compute the spatial distribution of long-term monthly mean surface air temperature $T$, and long-term monthly standard deviation of daily mean surface air temperature $\sigma$. Daily mean surface air temperature is computed in first hand as an average of the four daily analysis time-steps (00:00, 06:00, 12:00, 18:00). Standard deviation is calculated using temperature deviations relative to the long-term daily mean for the entire period, in order to remove variability associated with the annual cycle. Note that this is an improvement over our previous methods \citep{seguinot-2013,rogozhina-rau-inpress} that tended to overestimate spring and autumn $\sigma$ values.

\begin{figure*}
    \centering\includegraphics{stdev-param-scatter-sigma-era40-grl-all-large}
    \caption{Long-term monthly standard deviation compared to the long-term monthly mean of daily mean surface air temperature over the Greenland ice sheet, according to the ERA-40 reanalysis data \citep{uppala-etal-2005} over the 44-years period 1958–2001. Seasons are coloured in red (JJA), yellow (SON), blue (DJF) and green (MAM). The solid line corresponds to a linear regression over all points. Dashed contours represent the effect of daily variability on effective temperature for melt, also shown in the 3D wireframe inset.}
    \label{fig:grl}
\end{figure*}

Over the Greenland ice sheet, there exists an anticorrelation between the long-term monthly temperature means $T$ and the long-term monthly standard deviations $\sigma$ (Figure~\ref{fig:grl}). We then derive a linear relation between T and $\sigma$,

\begin{equation} \label{eq:sigma}
    \sigma = 0.15 \cdot T + 2.01.
\end{equation}

Although a significant spread exists around winter temperatures (Figure~\ref{fig:grl}, blue), the reconstructed trend appears more robust over the period of intensive surface melting (Figure~\ref{fig:grl}, red), when the standard deviation parameter exerts strongest influence on modelled melt rates \citep{rogozhina-rau-inpress}. To support this point we define the effective temperature for melt,

\begin{equation} \label{eq:teff}
    T_e = \frac{d\mathrm{PDD}}{dt}
        = \frac{\sigma}{\sqrt{2\pi}} \exp\left({-\frac{T_a^2}{2\sigma^2}}\right)
            + \frac{T_a}{2} \, \mathrm{erfc} \left(-\frac{T_a}{\sqrt{2}\sigma}\right).
\end{equation}

The net effect of daily variability on melt can then be expressed as a shift in effective temperature (Figure~\ref{fig:grl}, dashed contours),

\begin{equation} \label{eq:dteff}
    \Delta T_e = T_e - \max(T_a, 0),
\end{equation}

Note that $\Delta T_e$ is always positive, hence the use of higher values systematically results in increased modelled surface melt \citep{rogozhina-rau-inpress}. A decrease and associated restriction of ice melt in response to increasing air temperatures indicates a natural mechanism for self-inhibition of ice loss, which is, under the conditions close to the melting point, modulated by endothermic phase changes (Figure~\ref{fig:grl}).

The strong dependence of on air temperature shown by reanalysis data suggests that previously discussed relations between standard deviation and ice surface elevation \citep{fausto-etal-2009,fausto-etal-2011} originate from lapse-rate effects on near-surface air temperature and are only valid under present-day climatic conditions. In addition, a nearly linear dependence of $\sigma$ on temperature provides a simple recipe for a dynamic adjustment of the standard deviation parameter to evolving external forcing. Although the inferred relation is generally inapplicable to ice-free territories where more complex connections should be expected \citep{seguinot-2013}, our analysis does however reveal a similar relation between mean temperature and daily variability across Antarctica (Figure~\ref{fig:both}) and thus suggests potential applicability of a comparable parametrization to modelling studies of glaciations in regions with essentially different geographical and climatic settings.

\begin{figure}
    \centering\includegraphics{stdev-param-scatter-sigma-era40-both-zoom-07}
    \caption{July Greenland (red) and January Antarctic (gray) long-term monthly standard deviation compared to the long-term monthly mean of daily mean surface air temperature over the ice sheets, according to the ERA-40 reanalysis data \citep{uppala-etal-2005} over the 44-years period 1958–2001.}
    \label{fig:both}
\end{figure}

% ----------------------------------------------------------------------

\section{Acknowledgments}

DAAD, Ahlmann.

\bibliography{references}
\bibliographystyle{igs}

\end{document}
