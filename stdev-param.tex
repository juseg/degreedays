\documentclass[review]{igs}
\usepackage[utf8]{inputenc}
\usepackage{igsnatbib}
\usepackage{stfloats}
\usepackage{graphicx}
\graphicspath{{figures/}}

% ----------------------------------------------------------------------

\begin{document}

\title[Correspondence]{Correspondence}
\author[Seguinot and Rogozhina]{Julien Seguinot$^{1,2,3}$ and Irina Rogozhina$^1$}
\affiliation{%
    $^1$Helmholtz Centre Potsdam, GFZ German Research Centre for Geosciences,
    DE-14473 Potsdam, Germany\\
    $^2$Department of Physical Geography and Quaternary Geology,
    Stockholm University, SE-10691 Stockholm, Sweden\\
    $^3$Bolin Centre for Climate Research,
    Stockholm University, SE-10691 Stockholm, Sweden}
\abstract{Empty (no abstract for correspondence).}

\maketitle

% ----------------------------------------------------------------------

\emph{Daily temperature variability predetermined by thermal conditions over ice sheet surfaces}

~

Evolution of continental-scale ice sheets is affected by long-term changes in atmospheric conditions through shifts of the balance between surface accumulation and melt. Although surface melt primarily follows the annual temperature cycle, it is also significantly affected by occasional excursions of air temperature above the freezing point caused by daily weather variability \citep{arnold-mackay-1964}.

In multimillenial numerical simulations of ice sheet growth and decay, surface melt is commonly assumed proportional to the number of positive degree-days \citep[PDD;][]{hock-2003}. Daily variability is then typically included in the models under an assumption of normal temperature distribution, using a standard deviation parameter $\sigma$ in the PDD computation \citep{braithwaite-1984,reeh-1991,calov-greve-2005};

\begin{eqnarray} \label{eq:pdd}
    \mathrm{PDD} &=& \frac{1}{\sigma\sqrt{2\pi}}
        \int_{t_1}^{t_2} \mathrm{d}t
        \int_{0}^{\infty} \mathrm{d}T \,
        T \exp\left({-\frac{(T-T_{ann})^2}{2\sigma^2}}\right)\\
    &=& \int_{t_1}^{t_2} \mathrm{d}t
        \left[\frac{\sigma}{\sqrt{2\pi}} \exp\left({-\frac{T_{ann}^2}{2\sigma^2}}\right)
        + \frac{T_{ann}}{2} \, \mathrm{erfc} \left(-\frac{T_{ann}}{\sqrt{2}\sigma}\right)\right],
\end{eqnarray}

where t is the time, $T$ is the air temperature [$^\circ$C], and $\sigma$ is the standard deviation of $T$ [K] from the annual mean air temperature cycle $T_{ann}$ [$^\circ$C]. According to atmospheric reanalysis data, $\sigma$ is subject to significant variation over the globe, which has a significant influence on the response of surface mass balances \citep{seguinot-2013}.

Over the Greenland ice sheet, an analysis of in-situ temperature measurements indicates that the present-day distribution of $\sigma$ is subject to an annual variability, with lowest values occurring in the melting period and at the Greenland ice sheet margins, and can be linearly related to ice surface elevation \citep{fausto-etal-2009,fausto-etal-2011}. Including such seasonal and spatial $\sigma$ variability in simulations of the Greenland ice sheet is key to modelling the present-day ice surface response in agreement with observations \citep{rogozhina-rau-inpress}. However, the importance of a PDD approach is its applicability to paleoglaciological studies, requiring realistic $\sigma$ values under climate conditions different from today. Here, we present evidence from reanalysis data that the $\sigma$ distribution over the Greenland ice sheet is largely related to variations in near-surface air temperature, which provides the basis for a parametrization of daily variability over ice sheet surfaces.

Using the European Centre for Medium-Range Weather Forecasts’ ERA-40 reanalysis \citep{uppala-etal-2005} over a 44-years period 1958--2001, we compute the spatial distribution of long-term monthly mean surface air temperature $T_{ann}$, and long-term monthly standard deviation of daily mean surface air temperature $\sigma$. Daily mean surface air temperature is computed in first hand as an average of the four daily analysis time-steps (00:00, 06:00, 12:00, 18:00). Standard deviation is calculated using temperature deviations relative to the long-term daily mean for the entire period, in order to remove variability associated with the annual cycle. This approach is an improvement over our previous methods \citep{seguinot-2013,rogozhina-rau-inpress} which tended to overestimate spring and autumn $\sigma$ values.

\begin{figure*}
    \centering\includegraphics{stdev-param-scatter-sigma-era40-grl-all-large}
    \caption{Long-term monthly standard deviation $\sigma$ compared to the long-term monthly mean surface air temperature $T_ann$ over the Greenland ice sheet, according to the ERA-40 reanalysis \citep{uppala-etal-2005} over a 44-years period 1958–2001. Seasons are coloured in red (JJA), yellow (SON), blue (DJF) and green (MAM). The solid line corresponds to a $1/\sigma$-weighted least square regression over all data points. Dashed lines represent the effect of daily variability on effective temperature for melt $\Delta T_{eff}$. As shown by the 3D wireframe inset, $\Delta T_{eff}$ is always positive, and increases when $T_ann$ approaches the melting point.}
    \label{fig:grl}
\end{figure*}

Over the Greenland ice sheet, there exists an anticorrelation between long-term monthly temperature means $T_{ann}$ and long-term monthly standard deviations $\sigma$ (Figure~\ref{fig:grl}). Although a significant spread exists around winter temperatures (Figure~\ref{fig:grl}, blue), this spread becomes much more restricted when approaching conditions of intensive surface melting during summer (Figure~\ref{fig:grl}, red), when the standard deviation parameter exerts strongest influence on modelled melt rates \citep{rogozhina-rau-inpress}. Using a $1/\sigma$-weighted least square regression, we derive a linear relationship between $\sigma$ and $T_{ann}$,

\begin{equation} \label{eq:sigma}
    \sigma = 0.17 \cdot T_{ann} + 1.68\,.
\end{equation}

To emphasize conditions in which daily variability exerts strongest influence on modelled melt rates, we define the effective temperature for melt as

\begin{equation} \label{eq:teff}
    T_{eff} = \frac{d\mathrm{PDD}}{dt}
        = \frac{\sigma}{\sqrt{2\pi}} \exp\left({-\frac{T_{ann}^2}{2\sigma^2}}\right)
            + \frac{T_{ann}}{2} \, \mathrm{erfc} \left(-\frac{T_{ann}}{\sqrt{2}\sigma}\right).
\end{equation}

The net effect of daily variability $\sigma$ on melt can then be expressed as a shift in effective temperature (Figure~\ref{fig:grl}, dashed lines),

\begin{equation} \label{eq:dteff}
    \Delta T_{eff} = T_{eff} - \max(T_{ann}, 0),
\end{equation}

Note that $\Delta T_{eff}$ is always positive (Figure~\ref{fig:grl}, 3D inset), hence the use of higher values systematically results in increased modelled surface melt \citep{rogozhina-rau-inpress}. The decrease of $\sigma$ in response to increasing air temperatures tends to restrict ice surface melt, thereby indicating a natural mechanism for self-inhibition of ice loss from the ice sheet (Figure~\ref{fig:grl}).

The strong dependence of $\sigma$ on near-surface air temperature shown by the reanalysis data points out that previously discussed relationships between standard deviation and ice surface elevation \citep{fausto-etal-2009,fausto-etal-2011} originate from lapse-rate effects on near-surface air temperature and are only valid under present-day climatic conditions. However, a nearly linear dependence of $\sigma$ on near-surface air temperature provides a simple proxy for a dynamic adjustment of the standard deviation parameter to evolving external forcing. Although the inferred relationship does generally not hold true over ice-free territories, where more complex connections should be expected \citep{seguinot-2013}, our analysis does, however, reveal a similar relationship between near-surface mean air temperature and daily variability across Antarctica (Figure~\ref{fig:both}) and therefore indicates potential applicability of this parametrization to modelling studies of former glaciation in regions with essentially different geographical and climatic settings.

\begin{figure}
    \centering\includegraphics{stdev-param-scatter-sigma-era40-both-zoom-07}
    \caption{July Greenland (red) and January Antarctic (gray) long-term monthly standard deviation $\sigma$ compared to the long-term monthly mean surface air temperature $T$ over the ice sheets, according to ERA-40 reanalysis data \citep{uppala-etal-2005} over a 44-years period 1958–2001.}
    \label{fig:both}
\end{figure}

% ----------------------------------------------------------------------

\section{Acknowledgments}

Funding was provided by the German Academic Exchange Service (DAAD) and the Lillemor och Hans W:son Ahlmanns fond för geografisk forskning. This study is part of the multinational research initiative IceGeoHeat, and the Cordilleran ice sheet through a glacial cycle modelling study funded through A. Stroeven (Stockholm University).

\bibliography{references}
\bibliographystyle{igs}

\end{document}
