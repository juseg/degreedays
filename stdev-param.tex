\documentclass[review]{igs}
\usepackage[utf8]{inputenc}
\usepackage{igsnatbib}
\usepackage{stfloats}
\usepackage{graphicx}
\graphicspath{{figures/}}

% ----------------------------------------------------------------------

\begin{document}

\title[Correspondence]{Correspondence}
\author[Seguinot and Rogozhina]{Julien Seguinot$^{1,2}$ and Irina Rogozhina$^1$}
\affiliation{%
    $^1$Helmholtz Centre Potsdam, GFZ German Research Centre for Geosciences,
    DE-14473 Potsdam, Germany\\
    $^2$Department of Physical Geography and Quaternary Geology
    and the Bolin Centre for Climate Research,
    Stockholm University, SE-10691 Stockholm, Sweden}
\abstract{Empty (no abstract for correspondence).}

\maketitle

% ----------------------------------------------------------------------

\emph{Daily temperature variability predetermined by thermal conditions over ice sheet surfaces}

~

Evolution of continental-scale ice sheets is affected by long-term changes in atmospheric conditions through shifts of the balance between surface accumulation and melt. Although surface melt primarily follows the annual temperature cycle, it is also significantly affected by occasional excursions of near-surface air temperature above the freezing point caused by daily weather variability \citep{arnold-mackay-1964}.

In multimillenial numerical simulations of ice sheet growth and decay, surface melt is commonly assumed proportional to the number of positive degree-days \citep[PDD;][]{hock-2003}. Daily variability is then typically included in the models under an assumption of normal temperature distribution, using a standard deviation parameter $\sigma$ in the PDD computation \citep{braithwaite-1984,reeh-1991,calov-greve-2005};

\begin{eqnarray} \label{eq:pdd}
    \mathrm{PDD} &=& \frac{1}{\sigma\sqrt{2\pi}}
        \int_{t_1}^{t_2} \mathrm{d}t
        \int_{0}^{\infty} \mathrm{d}T \,
        T \exp\left({-\frac{(T-T_{ac})^2}{2\sigma^2}}\right)\\
    &=& \int_{t_1}^{t_2} \mathrm{d}t
        \left[\frac{\sigma}{\sqrt{2\pi}} \exp\left({-\frac{T_{ac}^2}{2\sigma^2}}\right)
        + \frac{T_{ac}}{2} \, \mathrm{erfc} \left(-\frac{T_{ac}}{\sqrt{2}\sigma}\right)\right],
\end{eqnarray}

where t is the time, $T$ is the near-surface daily mean air temperature [$^\circ$C], and $\sigma$ is the standard deviation of $T$ [K] from the annual temperature cycle $T_{ac}$ [$^\circ$C]. According to atmospheric temperature reanalyses, $\sigma$ is subject to significant variation over the globe, which has a significant influence on the response of surface mass balances \citep{seguinot-2013}.

Over the Greenland ice sheet, an analysis of in-situ temperature measurements indicates that the present-day distribution of $\sigma$ is subject to an annual variability, with lowest values occurring in the melting period and at the ice sheet margins, and can be linearly related to ice surface elevation \citep{fausto-etal-2009,fausto-etal-2011}. Including seasonally and spatially varied $\sigma$ in Greenland ice sheet modelling is crucial to obtaining better agreement with observations of the present-day ice surface response \citep{rogozhina-rau-2014}. However, the importance of a PDD approach is its applicability to paleoglaciological studies, requiring realistic $\sigma$ values under climate conditions different from today. Here, we present evidence from an atmospheric reanalysis that the $\sigma$ distribution over the Greenland ice sheet is largely related to variations in near-surface air temperature, which provides the basis for a parametrization of daily variability over ice sheet surfaces.

Using the 2-meter air temperature field from the European Centre for Medium-Range Weather Forecasts’ ERA-40 reanalysis \citep{uppala-etal-2005} over a 44-years period 1958--2001, we compute the spatial distribution of long-term monthly mean temperature $T_{m}$, and long-term monthly standard deviation of daily mean temperature $\sigma$. Daily mean temperature $T$ is computed in first hand as an average of the four daily analysis time-steps (00:00, 06:00, 12:00, 18:00). In order to remove variability associated with the annual cycle, monthly standard deviation $\sigma$ is calculated over the entire time period, after subtraction of the long-term daily mean component. This approach is an improvement over our previous methods \citep{seguinot-2013,rogozhina-rau-2014}, which included seasonal variability in the standard deviation calculation, and therefore tended to overestimate $\sigma$ values during spring and autumn, when the seasonal trend over a given month may be significant. Because we focus on temperature variability over ice sheet surfaces, ice-free grid cells are excluded from our analysis using the ERA-40 vegetation types and land-sea masks.

\begin{figure*}
    \centering\includegraphics{stdev-param-scatter-sigma-era40-grl-all-large}
    \caption{Long-term monthly standard deviation $\sigma$ versus long-term monthly mean near-surface air temperature $T_{m}$ over the Greenland ice sheet, according to the ERA-40 reanalysis \citep{uppala-etal-2005} over a 44-years period 1958–2001. Seasons are coloured in red (JJA), yellow (SON), blue (DJF) and green (MAM). The solid line corresponds to a $1/\sigma$-weighted least square regression over all data points (Eqn.~\ref{eq:sigma}). Dashed lines represent the effect of daily variability on effective temperature for melt $\Delta T_{eff}$ (Eqn.~\ref{eq:teff}). As shown by the 3D wireframe inset, $\Delta T_{eff}$ is always positive, and increases when $T_{m}$ approaches the melting point (Eqn. \ref{eq:dteff}).}
    \label{fig:grl}
\end{figure*}

Over the Greenland ice sheet, there exists an anticorrelation between long-term monthly temperature means $T_{m}$ and long-term monthly standard deviations $\sigma$ (Fig.~\ref{fig:grl}). Although a significant spread exists around winter temperatures (Fig.~\ref{fig:grl}, blue), this spread becomes much more restricted when approaching conditions of intensive surface melting during summer (Fig.~\ref{fig:grl}, red), when the standard deviation parameter exerts strongest influence on modelled melt rates \citep{rogozhina-rau-2014}. Using a $1/\sigma$-weighted least square regression, we derive a linear relationship between $\sigma$ and $T_{m}$ over the Greenland ice sheet,

\begin{equation} \label{eq:sigma}
    \sigma = -0.15 \cdot T_{m} + 1.66\,.
\end{equation}

To emphasize conditions in which daily variability has strongest influence on modelled melt rates, we define the effective temperature for melt as

\begin{equation} \label{eq:teff}
    T_{eff} = \frac{d\mathrm{PDD}}{dt}
        = \frac{\sigma}{\sqrt{2\pi}} \exp\left({-\frac{T_{m}^2}{2\sigma^2}}\right)
            + \frac{T_{m}}{2} \, \mathrm{erfc} \left(-\frac{T_{m}}{\sqrt{2}\sigma}\right).
\end{equation}

The net effect of daily variability $\sigma$ on melt can then be expressed as a shift in effective temperature (Fig.~\ref{fig:grl}, dashed lines),

\begin{equation} \label{eq:dteff}
    \Delta T_{eff} = T_{eff} - \max(T_{m}, 0),
\end{equation}

Note that $\Delta T_{eff}$ is always positive (Fig.~\ref{fig:grl}, 3D inset), hence the use of higher $\sigma$ values systematically results in increased modelled surface melt \citep{rogozhina-rau-2014}. The decrease of $\sigma$ in response to increasing temperature, most probably modulated by endothermic phase changes around the freezing point, tends to restrict ice surface melt, thereby indicating a natural mechanism for self-inhibition of ice loss from the ice sheet (Fig.~\ref{fig:grl}).

The strong dependence of $\sigma$ on near-surface air temperature shown by the reanalysis points out that previously discussed relationships between standard deviation and ice surface elevation \citep{fausto-etal-2009,fausto-etal-2011} originate from lapse-rate effects on temperature and are only valid under present-day climatic conditions. However, a nearly linear dependence of $\sigma$ on temperature provides a simple proxy for a dynamic adjustment of the standard deviation parameter to evolving climate forcing. Although the inferred relationship (Eqn. \ref{eq:sigma}) needs to be validated against observations, the observational record over the Greenland ice sheet is currently both too spatially scarce and too short to allow for the present analysis. It should also be noted that this relationship does generally not hold true over ice-free territories, where more complex connections should be expected \citep{seguinot-2013}. However, our analysis reveals that similar connections between daily variability and summer temperature exist over parts of Antarctica (Fig.~\ref{fig:both}). This indicates a potential for parametrization of daily temperature variability over glacierized regions with essentially different geographic and climatic setting than that of the Greenland ice sheet.

\begin{figure}
    \centering\includegraphics{stdev-param-scatter-sigma-era40-both-zoom-07}
    \caption{July Greenland (red) and January Antarctic (gray) long-term monthly standard deviation $\sigma$ versus long-term monthly mean near-surface air temperature $T$ over the ice sheets, according to ERA-40 reanalysis \citep{uppala-etal-2005} over a 44-years period 1958–2001.}
    \label{fig:both}
\end{figure}

% ----------------------------------------------------------------------

\section{Acknowledgements}

We would like to thank Arjen~P.~Stroeven, Qiong~Zhang and Johan~Kleman for their detailed suggestions on how to improve the manuscript before submission, Jo~Jacka for serving as a scientific editor, and two anonymous reviewers for supporting publication and spotting a deadly misprint. Funding was provided by the German Academic Exchange Service~(DAAD) grant No.~50015537, the Knut and Alice~Wallenberg Foundation, and the Lillemor and Hans~W:son~Ahlmanns fund for geographic research to J.~Seguinot, and by the Swedish Research Council~(VR) grant No.~2008-3449 to A.~P.~Stroeven. This study is part of the multinational research initiative IceGeoHeat.

\bibliography{references}
\bibliographystyle{igs}

\end{document}
