\documentclass{letter}

\signature{Julien Seguinot}
\address{
  julien.seguinot@natgeo.su.se\\\\
  Stockholm University\\
  SE-106 91 Stockholm\\
  SWEDEN}

\usepackage[utf8]{inputenc}
\usepackage{geometry}
\usepackage{xcolor}
\usepackage{letterbib}

\newcommand{\rev}[0]{\color{blue!50!black}\it}
\newcommand{\textrev}[1]{{\rev``#1''}}
\newcommand{\revpoint}[1]{{\rev\item``#1''}}
\newcommand{\todo}[1]{\textcolor{red!50!black}{Todo: #1}}
\newcommand{\done}[1]{\textcolor{green!50!black}{Done: #1}}

% ----------------------------------------------------------------------

\begin{document}
\begin{letter}{Reply to the second review by Roger Braithwaite}

\opening{Dear Editor,}

\textrev{As the person who introduced the variability of temperature ($\sigma$) to the glaciological world (Braithwaite, 1984), I take great pleasure in reviewing this letter. The standard deviation $\sigma$ describes the departure of daily temperatures from either (1) monthly mean temperature (Braithwaite, 1984) or (2) annual course of temperature following a sine wave (Reeh, 1991). The $\sigma$ parameter is used to estimate positive degree-day (PDD) totals to use in a simple glacier melt model. The author quotes several references (from 1991, 1997, 2005, 2012 and 2013) to show that such calculations are still being made, i.e. that the chosen value of $\sigma$ is still a live issue.}

I appreciate the reviewer's support in pointing out that the standard deviation formulations in PDD models are still a contemporary issue.

\textrev{What the author does is to estimate the variability of $\sigma$ over the whole world using a modern temperature data set and then comparing different PDD calculations. Results are presented in three colour-contoured maps, which represent very compact and concise presentations, well justifying the extra cost of colour printing. I am not quite sure why Figs 2 and 3 are done for a polar projection while Fig,. 1 is done for a Mercator (?) projection. Figs 2 and 3 might suit someone using PDD’s for sea ice in the Arctic Ocean while Fig. 1 would be more suitable for someone applying PDD’s to tropical or Southern Hemisphere glaciers. Would the author consider depositing a gridded geo-referenced table of the same results in an online depository?}

Although my figures use different geographical domains (Figure 1: global; Figure2: Arctic; new Figure 3: Greenland), all computations were made globally. I would be happy to share the result of these computations ($\sigma$, PDD and SMB) and put the data on-line in the IGS repository or elsewhere.

\textrev{My only suggestion for change is that the author should cite the letter by \citep{fausto-etal-2011} and discuss their results for Greenland within his claimed global and seasonal pattern. \citet{fausto-etal-2011} claim lower summer values of $\sigma$ over the Greenland ice sheet because surface temperatures are constrained by 0 $^\circ$C. I would be interested to know if the author’s ERA-temperature data pick up this effect that appears in observed data from Greenland.}

I thank the reviewer for pointing a reference highly relevant to my manuscript which I was not aware of. The values of standard deviation derived in this work exhibit a summer low consistently with \citet{fausto-etal-2011}. However this effect is not restricted to Greenland, but seem to apply in all regions that experience significant seasonality. This reference has been added to the manuscript, together with a comment on the standard deviation values obtained in this work.

\textrev{The above is a suggested change to further improve what is already an excellent piece of work.}

I thank Roger Braithwaite for this very supportive review.

\textbf{References}
\bibliography{../references}
\bibliographystyle{igs}

\closing{Best regards,}

\end{letter}
\end{document}
